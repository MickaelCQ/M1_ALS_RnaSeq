 Homéostasie de l’ARN
L’homéostasie de l’ARN désigne l’ensemble des mécanismes qui régulent la production, la maturation, le transport et la dégradation des ARN pour maintenir un équilibre fonctionnel dans la cellule.

 Transport axonal
Le transport axonal permet le déplacement des organites, protéines et ARN le long de l’axone, assurant la communication et la survie neuronale sur de longues distances.

 Autophagie
L’autophagie est un processus cellulaire de recyclage qui dégrade et élimine les composants cellulaires endommagés, contribuant à l’homéostasie et à la protection contre le stress cellulaire.
