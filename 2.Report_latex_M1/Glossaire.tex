\newglossaryentry{transport axonal}{
   name=Transport axonal,
   description={\glossaryhref{https://meshb.nlm.nih.gov/record/ui?ui=D001370}{Le transport axonal permet le déplacement des organites, protéines et ARN le long de l'axone en assurant la communication et la survie neuronale sur de longue distance. p}},
   sort=transport axonal}

\newglossaryentry{autophagie}{
   name=autophagie,
   description={\glossaryhref{https://meshb.nlm.nih.gov/record/ui?ui=D001343}{L’autophagie est un processus cellulaire de recyclage qui dégrade et élimine les composants cellulaires endommagés, contribuant à l’homéostasie et à la protection contre le stress cellulaire. p}},
   sort=autophagie}

\newglossaryentry{homeostasie}{
   name=homéostasie,
   description={\glossaryhref{https://meshb.nlm.nih.gov/record/ui?ui=D006706}{Ensemble des mécanismes qui régulent la production, la maturation, le transport et la dégradation pour maintenir un équilibre fonctionnel dans la cellule. p}},
   sort=homeostasie}

\newglossaryentry{apoptosis}{
    name=apoptose,
    description={\glossaryhref{https://meshb.nlm.nih.gov/record/ui?ui=D017209}{Processus de mort cellulaire programmée régulée. p}},
    sort=apoptose}

\newglossaryentry{transcriptome}{
    name=transcriptome,
    description={\glossaryhref{https://meshb.nlm.nih.gov/record/ui?ui=D059467}{Ensemble des ARN transcrits dans une cellule à un moment donné. p}},
    sort=transcriptome}

\newglossaryentry{prevalence}{
    name=prévalence,
    description={\glossaryhref{https://meshb.nlm.nih.gov/record/ui?ui=D015995}{Proportion d’individus dans une population donnée présentant une caractéristique (généralement une maladie) à un instant donné ou sur une période donnée. p}},
    sort=prevalence}

\newglossaryentry{haploinsuffisance}{
    name=haploinsuffisance,
    description={\glossaryhref{https://meshb.nlm.nih.gov/record/ui?ui=D057895}{Incapacité d'une seule copie fonctionnelle d’un gène à produire une quantité suffisante de produit génique (ARN ou protéine) pour assurer une fonction biologique normale, entraînant ainsi un phénotype pathologique. p}},
    sort=haploinsuffisance}
    
\newglossaryentry{paired-end}{
    name=séquençage pairé,
    description={\glossaryhref{https://www.illumina.com/science/technology/next-generation-sequencing/plan-experiments/paired-end-vs-single-read.html}{Méthode de séquençage où les deux extrémités d’un fragment d’ADN sont séquencées indépendamment, permettant d’obtenir deux lectures (lectures appariées) qui facilitent l’alignement et la détection des variants, notamment dans les régions complexes ou répétées. p}},
    sort=sequencage-paire}
