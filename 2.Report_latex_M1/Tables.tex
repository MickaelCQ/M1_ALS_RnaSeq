%-------------------------------------------------------------------------------------------------
%                                    TABLE DES ABBREVIATIONS ET DES SYMBOLES
%-------------------------------------------------------------------------------------------------x
\setlength{\nomitemsep}{-2pt} % Réduit l'espacement entre les items de nomenclature
% Table des symboles
%\nomenclature{{\large \textbf{$R_e$}}}{Lectures alignés sur l'échantillon \(e\)}
%\nomenclature{{\large \textbf{$F_e$}}}{Fragments alignés pour l'échantillon \(e\)}
%\nomenclature{{\large \textbf{$L_g$}}}{Longueur du gène d'intérêt \(g\)}
%\nomenclature{{\large \textbf{$\lambda_{x}$}}}{Facteur d'échelle de l'élément x}
%\nomenclature{{\large \textbf{$M_{g,e}$}}}{M-value pour le gène \(g\) dans l'échantillon \(e\)}
%\nomenclature{{\large \textbf{$A_{g,e}$}}}{A-value pour le gène \(g\) dans l'échantillon \(e\)}
%\nomenclature{{\large \textbf{$w_{g,e}$}}}{Poids basé sur la variance des M-values pour le gène \(g\) dans l'échantillon \(e\)}
%\nomenclature{{\large \textbf{$NR_{x}$}}}{Lectures Normalisées de l'élément x }
%\nomenclature{{\large \textbf{$\Pi$}}}{Produit}
%\nomenclature{{\large \textbf{$\sum$}}}{Somme}
%\nomenclature{{\large \textbf{$M_{g}$}}}{Moyenne géométrique du gène \(g\)}
%\nomenclature{{\large \textbf{$Q3_{g}$}}}{Troisième quartile des valeurs d'expression du gène \(g\)}
%\nomenclature{{\large \textbf{$Med_{x}$}}}{Médiane de l'élément x}
%\nomenclature{{\large \textbf{$k$}}}{Référence}
\printnomenclature


\section*{Liste des abréviations}

\begin{paracol}{2} % Création de 2 colonnes
    % Première colonne : Liste des abréviations
    \begin{center}
    \textbf{Acronymes}
    \end{center}
\begin{tabbing}
    \hspace{2cm} \= \kill % Pour définir la première colonne
    \textbf{\textsc{ADN}} \> Acide DésoxyRiboNucléique \\
    \textbf{\textsc{DGE}} \> Analyse d'Expression Différentielle\\
    \textbf{\textsc{ARN}} \> Acide RiboNucléique \\
    \textbf{\textsc{API}} \> Application Programming Interface \\
    \textbf{\textsc{CVS}} \> Concurrent Versions System \\
    \textbf{\textsc{FP}} \> Faux Positifs \\
    \textbf{\textsc{FN}} \> Faux Négatifs \\
    \textbf{\textsc{KB}} \> KiloBase \\
    \textbf{\textsc{SLA}} \> Sclérose Latérale Amyotrophique \\
    \textbf{\textsc{RCS}} \> Révision Control System \\
    \textbf{\textsc{SHD}} \> Séquençage Haut Débit \\
    \textbf{\textsc{SNP}} \> Polymorphisme nucléotidique unique \\
    \textbf{\textsc{SIF}} \> Singularity Image Format \\
    \textbf{\textsc{UML}} \> Language de modélisation unifié\\
    \textbf{\textsc{VP}} \> Vrai Positifs \\
    \textbf{\textsc{VN}} \> Vrai Négatifs \\
 
\end{tabbing}
    \hspace{5mm}
    \switchcolumn % Passe à la deuxième colonne
    % Deuxième colonne : Liste des symboles mathématiques
\begin{center}
    \textbf{Symboles}
\end{center}
\begin{tabbing}
    \hspace{1.5cm}  \= \kill
    $\mathcal{A}_{\text{x}}$ \> Alphabet de x \\
    $\Sigma_{\text{x}}$ \> Somme de x \\
    $Q_{P}$ \> Score de qualité Phred\\
    $Q_{A}$ \> Score de qualité Phred encodé en ASCII\\
    $S$ \> Séquence biologique\\
    $P$ \> Motif recherché \\
    $S_e$ \> Sensibilité \\
    $S_p$ \> Spécificité\\
    $\mathcal{T}$ \> Texte \\
    $w$ \> Mot\\
    $\mathcal{O}()$ \> Notation de Landau\\
    $\mathcal{SA}_{x}$ \> Table des suffixes de x\\
\end{tabbing}

\end{paracol}


\addcontentsline{toc}{section}{Liste des abréviations et symboles}

%-------------------------------------------------------------------------------------------------
%                                    TABLE DES FIGURES, TABLEAUX ET CODES
%-------------------------------------------------------------------------------------------------
\selectlanguage{french}
%____________________  Montage de la table des figures et des tableaux _______________

% Définir un nouvel environnement de flottant pour minted
\newfloat{listing}{htbp}{lop}
\floatname{listing}{Code}

% Redéfinir le titre de la liste des listings
\renewcommand{\lstlistlistingname}{Liste des codes}
\newpage

%-------------------------------------------------------------------------------------------------
%                                            TABLE DES MATIERES
%-------------------------------------------------------------------------------------------------

\tableofcontents
\vspace{10mm}
\newpage
